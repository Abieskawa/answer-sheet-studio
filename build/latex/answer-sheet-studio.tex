%% Generated by Sphinx.
\def\sphinxdocclass{report}
\documentclass[letterpaper,10pt,english]{sphinxmanual}
\ifdefined\pdfpxdimen
   \let\sphinxpxdimen\pdfpxdimen\else\newdimen\sphinxpxdimen
\fi \sphinxpxdimen=.75bp\relax
\ifdefined\pdfimageresolution
    \pdfimageresolution= \numexpr \dimexpr1in\relax/\sphinxpxdimen\relax
\fi
%% let collapsible pdf bookmarks panel have high depth per default
\PassOptionsToPackage{bookmarksdepth=5}{hyperref}
%% turn off hyperref patch of \index as sphinx.xdy xindy module takes care of
%% suitable \hyperpage mark-up, working around hyperref-xindy incompatibility
\PassOptionsToPackage{hyperindex=false}{hyperref}
%% memoir class requires extra handling
\makeatletter\@ifclassloaded{memoir}
{\ifdefined\memhyperindexfalse\memhyperindexfalse\fi}{}\makeatother

\PassOptionsToPackage{booktabs}{sphinx}
\PassOptionsToPackage{colorrows}{sphinx}

\PassOptionsToPackage{warn}{textcomp}

\catcode`^^^^00a0\active\protected\def^^^^00a0{\leavevmode\nobreak\ }
\usepackage{cmap}
\usepackage{xeCJK}
\usepackage{amsmath,amssymb,amstext}
\usepackage{babel}



\setmainfont{FreeSerif}[
  Extension      = .otf,
  UprightFont    = *,
  ItalicFont     = *Italic,
  BoldFont       = *Bold,
  BoldItalicFont = *BoldItalic
]
\setsansfont{FreeSans}[
  Extension      = .otf,
  UprightFont    = *,
  ItalicFont     = *Oblique,
  BoldFont       = *Bold,
  BoldItalicFont = *BoldOblique,
]
\setmonofont{FreeMono}[
  Extension      = .otf,
  UprightFont    = *,
  ItalicFont     = *Oblique,
  BoldFont       = *Bold,
  BoldItalicFont = *BoldOblique,
]



\usepackage[Sonny]{fncychap}
\ChNameVar{\Large\normalfont\sffamily}
\ChTitleVar{\Large\normalfont\sffamily}
\usepackage{sphinx}

\fvset{fontsize=\small,formatcom=\xeCJKVerbAddon}
\usepackage{geometry}


% Include hyperref last.
\usepackage{hyperref}
% Fix anchor placement for figures with captions.
\usepackage{hypcap}% it must be loaded after hyperref.
% Set up styles of URL: it should be placed after hyperref.
\urlstyle{same}


\usepackage{sphinxmessages}



\usepackage{fontspec}
\usepackage{xeCJK}

\newcommand{\sphinxsetcjkfonts}[1]{
  \setCJKmainfont{#1}
  \setCJKsansfont{#1}
  \setCJKmonofont{#1}
}

% Prefer Noto CJK on Read the Docs; fall back to common macOS fonts.
\IfFontExistsTF{Noto Sans CJK TC}{
  \sphinxsetcjkfonts{Noto Sans CJK TC}
}{
  \IfFontExistsTF{PingFang TC}{
    \sphinxsetcjkfonts{PingFang TC}
  }{
    \IfFontExistsTF{Noto Sans CJK SC}{
      \sphinxsetcjkfonts{Noto Sans CJK SC}
    }{
      \setCJKmainfont{FandolSong-Regular}
      \setCJKsansfont{FandolHei-Regular}
      \setCJKmonofont{FandolFang-Regular}
    }
  }
}

% Sphinx may write \selectlanguage*{...} into .aux/.toc; babel doesn't support
% the star-form, so make it accept and ignore the star.
\makeatletter
\@ifpackageloaded{babel}{
  \let\sphinx@selectlanguage\selectlanguage
  \renewcommand{\selectlanguage}{\@ifstar{\sphinx@selectlanguage}{\sphinx@selectlanguage}}
}{}
\makeatother


\title{Answer\sphinxhyphen{}Sheet\sphinxhyphen{}Studio}
\date{2026 年 01 月 11 日}
\release{0.0}
\author{Abieskawa and Calamus}
\newcommand{\sphinxlogo}{\vbox{}}
\renewcommand{\releasename}{發佈}
\makeindex
\begin{document}

\ifdefined\shorthandoff
  \ifnum\catcode`\=\string=\active\shorthandoff{=}\fi
  \ifnum\catcode`\"=\active\shorthandoff{"}\fi
\fi

\pagestyle{empty}
\sphinxmaketitle
\pagestyle{plain}
\sphinxtableofcontents
\pagestyle{normal}
\phantomsection\label{\detokenize{index::doc}}


\sphinxAtStartPar
\sphinxhref{https://github.com/Abieskawa/answer-sheet-studio/archive/refs/heads/main.zip}{點此下載程式(ZIP)} | {\hyperref[\detokenize{index:english}]{\sphinxcrossref{\DUrole{std,std-ref}{English}}}} | {\hyperref[\detokenize{index:zh-hant}]{\sphinxcrossref{\DUrole{std,std-ref}{繁體中文}}}}


\chapter{English}
\label{\detokenize{index:english}}\label{\detokenize{index:id1}}
\sphinxAtStartPar
Answer Sheet Studio lets teachers generate printable answer sheets and run local OMR recognition (no cloud upload):
\begin{itemize}
\item {} 
\sphinxAtStartPar
\sphinxstylestrong{Download page} – choose title, subject, class, number of questions (up to 100), and choices per question (ABC/ABCD/ABCDE). Generates a single\sphinxhyphen{}page PDF template (macOS/Windows compatible).

\item {} 
\sphinxAtStartPar
\sphinxstylestrong{Upload page} – drop a multi\sphinxhyphen{}page PDF scan. Recognition exports \sphinxcode{\sphinxupquote{results.csv}} (questions as rows, students as columns), \sphinxcode{\sphinxupquote{ambiguity.csv}} (blank/ambiguous/multi picks), plus \sphinxcode{\sphinxupquote{annotated.pdf}} that visualizes detections.

\item {} 
\sphinxAtStartPar
\sphinxstylestrong{Launcher} – double\sphinxhyphen{}click (\sphinxcode{\sphinxupquote{start\_mac.command}} or \sphinxcode{\sphinxupquote{start\_windows.vbs}}) to create a virtual env, install dependencies, start the local server, and open \sphinxcode{\sphinxupquote{http://127.0.0.1:8000}}.

\item {} 
\sphinxAtStartPar
The web UI supports \sphinxstylestrong{English} and \sphinxstylestrong{Traditional Chinese}. Use the language tabs in the header.

\end{itemize}


\section{Requirements}
\label{\detokenize{index:requirements}}\begin{itemize}
\item {} 
\sphinxAtStartPar
macOS ≥ 13 or Windows 11

\item {} 
\sphinxAtStartPar
Python \sphinxstylestrong{3.10+} (3.11 recommended; 3.10–3.13 supported). If you’re on Python 3.14 and installs fail, use Python 3.10–3.13.

\item {} 
\sphinxAtStartPar
On Windows, ensure “Add python.exe to PATH” during installation (and keep the \sphinxcode{\sphinxupquote{py}} launcher enabled if offered).

\item {} 
\sphinxAtStartPar
Internet access the first time to download Python packages (FastAPI, PyMuPDF, OpenCV, NumPy, etc.).

\end{itemize}


\section{Quick Start}
\label{\detokenize{index:quick-start}}

\subsection{macOS}
\label{\detokenize{index:macos}}\begin{enumerate}
\sphinxsetlistlabels{\arabic}{enumi}{enumii}{}{.}%
\item {} 
\sphinxAtStartPar
Double\sphinxhyphen{}click \sphinxcode{\sphinxupquote{start\_mac.command}} (or run \sphinxcode{\sphinxupquote{chmod +x start\_mac.command}} once if prompted).

\item {} 
\sphinxAtStartPar
First run creates \sphinxcode{\sphinxupquote{.venv}} and installs requirements; later runs reuse the existing \sphinxcode{\sphinxupquote{.venv}} (unless requirements changed).

\item {} 
\sphinxAtStartPar
Your browser opens \sphinxcode{\sphinxupquote{http://127.0.0.1:8000}}. Close the browser when you’re done; the server auto\sphinxhyphen{}exits after a period of inactivity.

\end{enumerate}


\subsection{Windows 11}
\label{\detokenize{index:windows-11}}\begin{enumerate}
\sphinxsetlistlabels{\arabic}{enumi}{enumii}{}{.}%
\item {} 
\sphinxAtStartPar
Install Python 3.10+ (3.11 recommended; 3.10–3.13 supported) from python.org and check “Add python.exe to PATH”.

\item {} 
\sphinxAtStartPar
Double\sphinxhyphen{}click \sphinxcode{\sphinxupquote{start\_windows.vbs}}.

\item {} 
\sphinxAtStartPar
First run installs requirements; later runs reuse the existing \sphinxcode{\sphinxupquote{.venv}} (unless requirements changed). If Windows Defender prompts for network access, allow it so the server can bind to localhost.

\end{enumerate}


\section{Important Notes}
\label{\detokenize{index:important-notes}}\begin{itemize}
\item {} 
\sphinxAtStartPar
For recognition, the \sphinxstylestrong{number of questions} and \sphinxstylestrong{choices per question} must match the generated answer sheet.

\item {} 
\sphinxAtStartPar
For more stable results: use a darker pen/pencil, scan at \sphinxstylestrong{300dpi}, and avoid skew/rotation.

\end{itemize}


\section{Output Files}
\label{\detokenize{index:output-files}}
\sphinxAtStartPar
After recognition, files are written under \sphinxcode{\sphinxupquote{outputs/<job\_id>/}}:
\begin{itemize}
\item {} 
\sphinxAtStartPar
\sphinxcode{\sphinxupquote{results.csv}}

\item {} 
\sphinxAtStartPar
\sphinxcode{\sphinxupquote{ambiguity.csv}}

\item {} 
\sphinxAtStartPar
\sphinxcode{\sphinxupquote{annotated.pdf}}

\item {} 
\sphinxAtStartPar
\sphinxcode{\sphinxupquote{input.pdf}} (original upload)

\end{itemize}


\section{Updating}
\label{\detokenize{index:updating}}\begin{itemize}
\item {} 
\sphinxAtStartPar
Open \sphinxcode{\sphinxupquote{http://127.0.0.1:8000/update}} and upload the latest ZIP (from GitHub Releases or \sphinxcode{\sphinxupquote{main.zip}}). The app will restart automatically.

\end{itemize}


\section{Debug Mode}
\label{\detokenize{index:debug-mode}}\begin{itemize}
\item {} 
\sphinxAtStartPar
Open \sphinxcode{\sphinxupquote{http://127.0.0.1:8000/debug}} and enter the Job ID (the folder name under \sphinxcode{\sphinxupquote{outputs/}}) to download diagnostic files (including \sphinxcode{\sphinxupquote{ambiguity.csv}}).

\end{itemize}


\section{Troubleshooting}
\label{\detokenize{index:troubleshooting}}\begin{itemize}
\item {} 
\sphinxAtStartPar
\sphinxstylestrong{pip install failed} – Check the launcher log; ensure network access and Python 3.10+. If you’re on Python 3.14, try Python 3.10–3.13.

\item {} 
\sphinxAtStartPar
\sphinxstylestrong{Port already in use / permission denied} – The launcher checks port availability. If macOS firewall blocks Python, allow incoming connections in System Settings > Network > Firewall.

\item {} 
\sphinxAtStartPar
\sphinxstylestrong{Server not opening} – Use the launcher log to identify crashes; you can also run \sphinxcode{\sphinxupquote{python run\_app.py}} inside \sphinxcode{\sphinxupquote{.venv}} manually for debugging.

\end{itemize}


\chapter{繁體中文}
\label{\detokenize{index:zh-hant}}\label{\detokenize{index:id2}}
\sphinxAtStartPar
Answer Sheet Studio 讓老師可以產生可列印的答案卡,並在本機進行 OMR 辨識(不上傳到雲端):
\begin{itemize}
\item {} 
\sphinxAtStartPar
\sphinxstylestrong{下載答案卡}:輸入標題、科目、班級、題數(最多 100 題)、每題選項(ABC / ABCD / ABCDE),產生單頁 PDF。

\item {} 
\sphinxAtStartPar
\sphinxstylestrong{上傳辨識}:上傳多頁 PDF(每頁一位學生)。會輸出 \sphinxcode{\sphinxupquote{results.csv}} (題號為列、學生為欄)、\sphinxcode{\sphinxupquote{ambiguity.csv}} (空白/模稜兩可/多選)、以及 \sphinxcode{\sphinxupquote{annotated.pdf}} (標註辨識結果)。

\item {} 
\sphinxAtStartPar
\sphinxstylestrong{Launcher}:雙擊啟動器(\sphinxcode{\sphinxupquote{start\_mac.command}} 或 \sphinxcode{\sphinxupquote{start\_windows.vbs}})即可建立虛擬環境、安裝套件、啟動伺服器並開啟 \sphinxcode{\sphinxupquote{http://127.0.0.1:8000}}。

\item {} 
\sphinxAtStartPar
網頁介面支援 \sphinxstylestrong{English / 繁體中文},可用頁首的語言切換。

\end{itemize}


\section{系統需求}
\label{\detokenize{index:id3}}\begin{itemize}
\item {} 
\sphinxAtStartPar
macOS 13+ 或 Windows 11

\item {} 
\sphinxAtStartPar
Python 3.10+(建議 3.11;支援 3.10–3.13)

\item {} 
\sphinxAtStartPar
Windows 安裝 Python 時請勾選「Add python.exe to PATH」(並保留 \sphinxcode{\sphinxupquote{py}} launcher)

\item {} 
\sphinxAtStartPar
第一次安裝需要網路下載 Python 套件(FastAPI、PyMuPDF、OpenCV、NumPy 等)

\end{itemize}


\section{快速開始}
\label{\detokenize{index:id4}}

\subsection{macOS}
\label{\detokenize{index:id5}}\begin{enumerate}
\sphinxsetlistlabels{\arabic}{enumi}{enumii}{}{.}%
\item {} 
\sphinxAtStartPar
雙擊 \sphinxcode{\sphinxupquote{start\_mac.command}} (若提示權限,先執行一次 \sphinxcode{\sphinxupquote{chmod +x start\_mac.command}})。

\item {} 
\sphinxAtStartPar
第一次會建立 \sphinxcode{\sphinxupquote{.venv}} 並安裝依賴套件;之後會重用既有 \sphinxcode{\sphinxupquote{.venv}} (除非 \sphinxcode{\sphinxupquote{requirements.txt}} 有變更)。

\item {} 
\sphinxAtStartPar
瀏覽器會自動開啟 \sphinxcode{\sphinxupquote{http://127.0.0.1:8000}}。用完關閉瀏覽器即可;伺服器會在一段時間無操作後自動結束。

\end{enumerate}


\subsection{Windows 11}
\label{\detokenize{index:id6}}\begin{enumerate}
\sphinxsetlistlabels{\arabic}{enumi}{enumii}{}{.}%
\item {} 
\sphinxAtStartPar
從 python.org 安裝 Python 3.10+(建議 3.11;支援 3.10–3.13),並勾選「Add python.exe to PATH」。

\item {} 
\sphinxAtStartPar
雙擊 \sphinxcode{\sphinxupquote{start\_windows.vbs}}。

\item {} 
\sphinxAtStartPar
第一次會安裝依賴套件;之後會重用既有 \sphinxcode{\sphinxupquote{.venv}} (除非 \sphinxcode{\sphinxupquote{requirements.txt}} 有變更)。若 Windows Defender 詢問是否允許網路連線,請允許(只會綁定 localhost)。

\end{enumerate}


\section{使用注意事項}
\label{\detokenize{index:id7}}\begin{itemize}
\item {} 
\sphinxAtStartPar
進行辨識時,\sphinxstylestrong{題數} 與 \sphinxstylestrong{每題選項(ABC/ABCD/ABCDE)} 必須與答案卡一致。

\item {} 
\sphinxAtStartPar
想要結果更穩定:建議用較深的筆、掃描 \sphinxstylestrong{300dpi},並避免歪斜/旋轉。

\end{itemize}


\section{輸出檔案}
\label{\detokenize{index:id8}}
\sphinxAtStartPar
辨識完成後,檔案會寫入 \sphinxcode{\sphinxupquote{outputs/<job\_id>/}}:
\begin{itemize}
\item {} 
\sphinxAtStartPar
\sphinxcode{\sphinxupquote{results.csv}}

\item {} 
\sphinxAtStartPar
\sphinxcode{\sphinxupquote{ambiguity.csv}}

\item {} 
\sphinxAtStartPar
\sphinxcode{\sphinxupquote{annotated.pdf}}

\item {} 
\sphinxAtStartPar
\sphinxcode{\sphinxupquote{input.pdf}} (原始上傳檔)

\end{itemize}


\section{更新}
\label{\detokenize{index:id9}}\begin{itemize}
\item {} 
\sphinxAtStartPar
開啟 \sphinxcode{\sphinxupquote{http://127.0.0.1:8000/update}},上傳最新 ZIP(GitHub Releases 或 \sphinxcode{\sphinxupquote{main.zip}})。更新後會自動重新啟動。

\end{itemize}


\section{Debug Mode(回報問題用)}
\label{\detokenize{index:id10}}
\sphinxAtStartPar
一般使用者不需要使用 Debug Mode。
\begin{itemize}
\item {} 
\sphinxAtStartPar
開啟 \sphinxcode{\sphinxupquote{http://127.0.0.1:8000/debug}},輸入 Job ID(\sphinxcode{\sphinxupquote{outputs/}} 底下的資料夾名稱)。

\item {} 
\sphinxAtStartPar
下載 \sphinxcode{\sphinxupquote{results.csv}}、\sphinxcode{\sphinxupquote{ambiguity.csv}}、\sphinxcode{\sphinxupquote{annotated.pdf}} (必要時再下載 \sphinxcode{\sphinxupquote{input.pdf}}),提供給開發者協助排查。

\end{itemize}


\section{排除問題}
\label{\detokenize{index:id11}}\begin{itemize}
\item {} 
\sphinxAtStartPar
\sphinxstylestrong{pip install failed} – 檢查 launcher log,確認有網路、Python 版本為 3.10+;若你是 Python 3.14,建議改用 3.10–3.13。

\item {} 
\sphinxAtStartPar
\sphinxstylestrong{Port already in use / permission denied} – 啟動器會檢查 port;若 macOS 防火牆阻擋 Python,請到 System Settings > Network > Firewall 放行。

\item {} 
\sphinxAtStartPar
\sphinxstylestrong{Server not opening} – 先看 launcher log 找出錯誤;也可以在 \sphinxcode{\sphinxupquote{.venv}} 裡直接跑 \sphinxcode{\sphinxupquote{python run\_app.py}} 方便除錯。

\end{itemize}



\renewcommand{\indexname}{索引}
\printindex
\end{document}